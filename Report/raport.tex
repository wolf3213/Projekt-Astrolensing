\documentclass[a4paper,11pt]{article}
\usepackage{latexsym}
\usepackage{hyperref}
\usepackage{graphicx}
\usepackage[MeX]{polski}
\usepackage[utf8]{inputenc}
\usepackage{geometry}
\usepackage{hyperref}
\usepackage{subcaption}
\usepackage{natbib}
\usepackage[nottoc,numbib]{tocbibind}
 \geometry{
 a4paper,
 left=20mm,
 top=30mm,
 }
\author{Adam Gonstal, Kamil Kolasa, Rafał Kornel, Konrad Maliszewski, Anna Olechowska}
\title{Poszukiwanie mikrosoczewek grawitacyjnych}
\frenchspacing
\newcommand{\ak}{\hspace{0.7 cm}}
\renewcommand{\abstractname}{Abstrakt}
\bibliographystyle{plainnat}
\setcitestyle{authoryear,open={(},close={)}}
\begin{document}
\maketitle
\newpage
\tableofcontents
\newpage
\begin{abstract}
\nocite{*}
\ak Poniżej opisany projekt studencki polegał na analizie fragmentu danych z projektu OGLE III w celu znalezienia zjawisk mikrosoczewkowania grawitacyjnego.  Autorom zostały udostępnione dane z teleskopu w Las Campanas w Chile, dotyczące m.in. pomiarów jasności dla ok. $260$ tysięcy  gwiazd, zbieranych na przestrzeni ok. $6$ lat. W ramach projektu utworzony został algorytm analizujący dane dla każdej gwiazdy i zwracający wykresy zależności jasności od czasu dla tych gwiazd, które według algorytmu mogły dawać efekt soczewki. Około $x\%$ zwróconych gwiazd okazało się rzeczywistymi przypadkami mikrosoczewkowania grawitacyjnego. Ponadto, $y$ ze znalezionych przez algorytm soczewek nie zostały zidentyfikowane przez zespół projektu OGLE III.
\end{abstract}
\section{Wstęp}
\subsection{Wstęp teoretyczny}
\ak Zjawisko soczewkowania grawitacyjnego wynika z zakrzywienia czasoprzestrzeni przez masy znajdujące się w niej. Konsekwencją tego jest poruszanie się promieni świetlnych po zakrzywionych torach, tj. najkrótszych możliwych, w przestrzeni Mińkowskiego. W związku z tym, w sytuacji gdy w okolicach linii łączącej źródło światła (np. galaktykę) z obserwatorem znajdzie się odpowiednio duża masa, światło biegnie omijając taką masę, co przedstawia Rysunek \ref{Fig_1}. 

\begin{figure}[h]
\centering
\includegraphics[width=0.9\textwidth]{Fig/Lens.jpeg}
\label{Fig_1}
\caption{Przykład zjawiska soczewkowania, gdzie źródło S' jest obserwowane jako dwa obrazy I+ i I- \citet{Lens}.}
\end{figure}

\ak Obraz źródła widziany przez obserwatora może ulec różnym deformacjom, tj. rozciągnięciu, przemieszczeniu, kilkukrotnemu odbiciu, a także wzmocnieniu, co w przypadku tego projektu jest najistotniejszym aspektem soczewkowania. Przykłady takiego zjawiska przedstawiają Rysunki \ref{Fig_2} i \ref{Fig_3}.
\newpage%UWAGA NEWPAGE
\begin{figure}[h]
\begin{subfigure}{0.5\textwidth}
\centering
\includegraphics[width=0.9\linewidth,height=5cm]{Fig/Horseshoe.jpeg}
\caption{Galaktyka LRG 3-757 soczewkująca obraz galaktyki znajdującej się za nią \citet{Horseshoe}}
\label{Fig_2}
\end{subfigure}
\hspace{0.5cm}
\begin{subfigure}{0.5\textwidth}
\centering
\includegraphics[width=0.9\linewidth,height=5cm]{Fig/Cross.jpeg}
\caption{Kwazar Q2237+030 soczewkowany przez galaktykę ZW 2237+030, tzw. krzyż Einsteina \citet{Cross}}
\label{Fig_3}
\end{subfigure}
\caption{Przykłady deformacji obrazu spowodowane soczewkowaniem grawitacyjnym}
\end{figure}
\ak W szczególnych przypadkach, gdy masa soczewkująca jest stosunkowo nieduża, a jej tor ruchu przecina bądź jest bardzo bliski torowi promieni świetlnych od źródła do obserwatora, efekty deformacji obrazu mają zbyt małe rozmiary kątowe, by udało się je zaobserwować z Ziemi. W takich sytuacjach jedyną obserwowalną konsekwencją zajścia soczewki jest wzmocnienie jasności. Ten specyficzny rodzaj soczewkowania nazywany jest mikrosoczewkowaniem grawitacyjnym. Przykładem jego może być obiekt z pobliskiej galaktyki wysyłający ku Ziemi promieniowanie elektromagnetyczne, na którego drodze znajduje się masywna planeta. Wzmocnienie można wyrazić jako wielkość $\mu$, będącą ilorazem strumienia światła bez wzmocnienia oraz z wzmocnieniem. Ponieważ natężenie światła $I$ jest stałe w czasie, będzie to wyłącznie iloraz kątów bryłowych, z których światło dociera do  obserwatora.\\
\begin{equation}
\centering
\mu=\frac{Id\Omega}{Id\Omega_{0}}=\frac{d\Omega}{d\Omega_{0}}
\label{Eq_1}
\end{equation}
\flushleft
\ak Wzmocnienie można również dobrze opisać za pomocą odległości $u$ źródła światła od soczewki, którą można opisać za pomocą kilku parametrów, które dla danej soczewki można przyjąć jako stałe w trakcie trwania zjawiska. Wspomniane parametry geometryczne mają wpływ na $t_{E}$, tj. czas Einsteina. Wielkość $b$ jest wielkością analogiczną do parametru zderzenia i także jest stała. Czas $t_{0}$ jest momentem największego wzmocnienia, z kolei jedyną zmienną we wzorze \ref{Eq_2} jest czas {t}.
\begin{equation}
\centering
u(t)=\sqrt{\left(\frac{t-t_{0}}{t_{E}}\right)^{2}+b^{2}}
\label{Eq_2}
\end{equation}
\ak Znając już zależność $u(t)$ można powiązać ją ze wspomnianym wcześniej wzmocnieniem $\mu$, tj. wyprowadzić wzór \ref{Eq_3}, zwany także krzywą Paczyńskiego. Przykładową krzywą Paczyńskiego przedstawia Rysunek \ref{Fig_997}. %tutaj ref do krzywej w subsection "Projekt OGLE"
W skali sześciu lat mikrosoczewka trwająca ok. $70$ dni widocznie wyróżnia się skokiem jasności w trakcie trwania zjawiska, co było punktem wyjściowym przy konstrukcji algorytmu i zostanie opisane dokładniej w kolejnych rozdziałach.
\begin{equation}
\centering
\mu(u)=\frac{u^{2}+2}{u\sqrt{u^{2}+4}}
\label{Eq_3}
\end{equation}
\flushleft
\subsection{Projekt OGLE}

\ak Projekt OGLE  tj. ,,Optical Gravitational Lensing Experiment'' jest projektem naukowym prowadzonym w obserwatorium Las Campanas w Chile, przez Obserwatorium Astronomiczne Uniwersytetu Warszawskiego, w ramach którego m.in. wykonywane są pomiary jasności gwiazd, mające na celu poszukiwanie zjawisk mikrosoczewkowania. 

\ak Projekt kierowany jest przez prof. Andrzeja Udalskiego, a jego zasadniczym obszarem obserwacji są rejony w pobliżu centrum naszej galaktyki. Od początku istnienia projektu (kwiecień 1992 roku) zaobserwowano: 20 nowych planet pozasłonecznych, ponad 4000 zjawisk mikrosoczewkowania oraz kilkaset tysięcy nowych gwiazd zmiennych\footnote{Dane ze strony internetowej projektu: \textit{http://www.astrouw.edu.pl/index.php/ogle-artykul}}. Projekt jest aktualnie (od marca 2010 roku) w czwartej fazie realizacji. Poprzednie fazy miały miejsce w latach: 1992-1995 (OGLE-I), 1997-2001 (OGLE-II), 2001-2009 (OGLE-III) i wiązały się ze stosowaniem innych detektorów.

\ak Dane na których oparty jest nasz projekt studencki pochodzą z projektu OGLE-III i zawierają \bf{X} plików w, którym jest średnio \bf{Y} pomiarów jasności w przeciągu 6 lat. 
\section{Analiza danych}

\section{Rezultaty}
\bibliography{bibliografia}
\end{document}
